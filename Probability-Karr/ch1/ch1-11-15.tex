% Exercise 1.11
\begin{exercise}
  Let $A, B$ and $C$ be disjoint events with $P(A)=.6$, $P(B)=.3$ and $P(C)=.1$. Calculate the probabilities of all events in the $\sigma$-algebra generated by $\{ A, B, C \} $.
\end{exercise}
\begin{solution}
  Given $A, B, C$ are disjoint events, $\sigma(\{ A, B, C \} ) = \{ \emptyset, A, B, C, A\cup B, A\cup C, B\cup C, A\cup B\cup C \} $. Then
  \begin{align*}
    P(\emptyset) &= 0 \\
    P(A) &= .6 \\
    P(B) &= .3 \\
    P(C) &= .1 \\
    P(A\cup B) &= P(A) + P(B) = .6 + .3 = .9 \\
    P(A\cup C) &= P(A) + P(C) = .6 + .1 = .7  \\
    P(B\cup C) &= P(B) + P(C) = .3 + .1 = .4  \\
    P(A\cup B\cup C) &= P(A) + P(B) + P(C) = .9 + .6 + .3 = 1 \\
  \end{align*}
\end{solution}


% Exercise 1.12
\begin{exercise}
  Verify that (1.8) defines a probability.
\end{exercise}
\begin{solution}
  To verify $\varepsilon_\omega$ (the point mass at $\omega\in\Omega$) is a probability, we need to verify the three conditions in the Definition 1.19. \\
  a) As $\varepsilon_\omega$ has values in $\{ 0, 1 \} $, $\varepsilon_\omega(A)\geq 0$ for all $A\in\cal F$. \\
  b) As $\omega\in\Omega$, $\varepsilon_\omega(\Omega) = 1$. \\
  c) Let $A_1, A_2, \ldots$ be disjoint sets in $\cal F$. It is either $\omega\in\Sigma_{n=1}^\infty A_n$ or $\omega\notin\cup_{n=1}^\infty A_n$. \\
  In the former case, $\varepsilon_\omega(\Sigma_{n=1}^\infty A_n) = 1$, and $\omega$ must exist in only $A_k$ for some $k\geq 1$, so $\Sigma_{n=1}^\infty P(A_n) = P(A_k) + \Sigma_{n\neq k} P(A_n) = 1 + 0 = 1$.
\end{solution}


% Exercise 1.13
\begin{exercise}
  Confirm that (1.9) defines a probability.
\end{exercise}
\begin{solution}
  a) Since $p_i$ are defined to be positive, $\Sigma_{i\in I}p_i\geq 0$ for all $I\subseteq \{ 1,\ldots, n \} $. \\
  b) By definition of $p_i$, $\Sigma_{i=1}^n p_i=1$. \\
  c) The countable distjoint sets can only be finite in this problem. By definition of $p_i$, c) can be easily verified.
\end{solution}


% Exercise 1.14
\begin{exercise}
  Prove that  for all events $A$ and $B$,
  \begin{align*}
    P(A\Delta B) &= P(A\cup B) - P(A\cap B) \\
      &= P(A) + P(B) - 2P(A\cap B).
  \end{align*}
\end{exercise}
\begin{solution}
  For the first equality, since $A\cup B = A\Delta B + A\cup B$, the sum of two disjoint sets, we have $P(A\cup B) = P(A\Delta B) + P(A\cup B)$. Simply rearranging gives the first equality.

  For the second equality, given $A = A\backslash B + A\cap B$ and $B = B\backslash A + A\cap B$, we can get $P(A\backslash B) = P(A) - P(A\cap B)$ and $P(B\backslash A) = P(B) - P(A\cap B)$. Since $A\Delta B = A\backslash B + B\backslash A$, we have $P(A\Delta B) = P(A\backslash B) + P(B\backslash A) = P(A) + P(B) - 2P(A\cap B)$.
\end{solution}


% Exercise 1.15
\begin{exercise}
  Let $\Omega$ be a finite set and let $\cal F=\cal P(\Omega)$ be the $\sigma$-algebra of all subsets of $\Omega$. For $A\subseteq \Omega$, let $|A|$ be the number of elements in $A$. Prove that the formula $P(A)=|A|/|\Omega|$ defines a probability on $(\Omega, \cal F)$ (the uniform distribution on $\Omega$).
\end{exercise}
\begin{solution}
  Assume $\Omega$ is not empty. \\
  a) Since $|A|\geq 0$ and $|\Omega|\geq 1$, $P(A)\geq 0$ for all $A\in \cal F$. \\
  b) $P(\Omega) = |\Omega| / |\Omega| = 1$. \\
  c) Let $A_1, A_2, \ldots, A_n$ be a finitne number of disjoint sets in $\cal F$. The number of elements in $\Sigma_{k=1}^n A_k$ must be equal to the total number of elements in $A_k$ for all $k\in\{ 1,\ldots, n \} $, i.e. $|\Sigma_{k=1}^n A_k| = \Sigma_{k=1}^n|A_k|$. Thus, c) can be easily verified.
\end{solution}
