% Exercise 2.6
\begin{exercise}
  Show that if $X$ is discrete with values in the countable set $C$, then for every $B\in\cB(\bR)$, $P\{X\in B\}=\Sigma_{a\in C\cap B}P\{X=a\}$.
\end{exercise}
\begin{solution}
  We can write $B=(C+C^c)\cap B=C\cap B+C^c\cap B$. Since $X$ is discrete with values only in $C$, $P\{X\in C^c\cap B\}=0$ and $P\{X\in B\} = P\{X\in C\cap B\}$. As $C$ is countable, then $C\cap B = \cup_{a\in C\cap B}\{a\}$ is also countable. Hence
  \[ P\{X\in B\} = P\{X\in\cup_{a\in C\cap B}\{a\}\} = \Sigma_{a\in C\cap B}P\{X=a\} \]
\end{solution}


% Exercise 2.7
\begin{exercise}
  Consider a random permutation of the intergers $\{1,\ldots,n\}$, with all $n!$ permutations equally likely. For each $i$, let $X_i$ be the integer in the $i$th position, and let $A_i=\{X_i=i\}$. (Physically, this means that there is a \textit{match} in the $i$th position.) \\
  a) Use the inclusion-exclusion principle (Exercise 1.25) to show that the probability of at least one match is
  \[ P(\cup_{i=1}^nA_i) = \Sigma_{k=1}^n (-1)^{k-1}/k!. \]
  b) Let $p_n$ be the probability of no match. Show that $\lim_n p_n=1/e$.
\end{exercise}
\begin{solution}
  a) Recall the inclusion-exclusion principle:
  \[ P(\cup_{i=1^n}A_i) = \Sigma_{k=1}^n(-1)^{k-1}S_k \]
  where $S_k = \Sigma_{|J|=k} P(\cap_{j\in J}A_j)$ and $J\subseteq\{1,\ldots,n\}$. As $P(\cap_{j\in J}A_j)$ is the probability of there is a match in the $j$th position for every $j\in J$, we can derive it by counting. Consider the $k$ positions in $J$, the first position has $n$ intergers to choose, and the second position then has $n-1$ integers to choose, etc. Hence
  \[ P(\cap_{j\in J}A_j) = \frac{1}{n(n-1)\cdots(n-k+1)} = \frac{(n-k)!}{n!} .\]
  For $|J|=k$, there are $\binom{n}{k}$ numbers of distinct $J$. Thus
  \[ S_k = \Sigma_{|J|=k}P(\cap_{j\in J}A_j) = \binom{n}{k}\frac{(n-k)!}{n!} = \frac{n!}{k!(n-k)!}\cdot\frac{(n-k)!}{n!} = \frac{1}{k!} .\]

  b)\begin{align*}
    p_n = 1 - P(\cup_{i=1}^n A_i) &= 1 - \Sigma_{k=1}^n\frac{(-1)^{k-1}}{k!} \\
      &= 1 + \Sigma_{k=1}^n\frac{(-1)^{k}}{k!} \\
      &= \Sigma_{k=0}^n\frac{(-1)^{k}}{k!} \to \frac{1}{e} \quad\mbox{as $n\to\infty$} .
  \end{align*}
  The limit is justified by the Taylor expansion of $e^{-1}$.
\end{solution}


% Exercise 2.8
\begin{exercise}
  Let $X$ have distribution $P(\lambda)$. Show that the function $i\mapsto P\{X=i\}$ is first increasing and then descreasing, with its maximum value at $\lfloor\lambda\rfloor$, the largest integer less than or equal to $\lambda$.
\end{exercise}
\begin{solution}
  Let us first assume that $\lambda$ is not an integer. For $k\geq 1$, let $P\{X=k\} - P\{X=k-1\} > 0$, that is
  \[ P\{X=k\} - P\{X=k-1\} = \frac{e^{-\lambda}\lambda^{k}}{k!} - \frac{e^{-\lambda}\lambda^{k-1}}{(k-1)!} = \frac{e^{-\lambda}\lambda^{k-1}}{(k-1)!}\big(\frac{\lambda}{k} -1\big) > 0 .\]
  As $\frac{e^{-\lambda}\lambda^{k-1}}{(k-1)!}>0$, this gives $\frac{\lambda}{k}-1> 0$ and $k<\lambda$, which shows that the function $i\mapsto P\{X=i\}$ is increasing for $k<\lambda$. Next, we let $P\{X=k\} - P\{X=k+1\} > 0$ for $k\geq 0$. Similarly, we get $k > \lambda - 1$, which shows that $i\mapsto P\{X=i\}$ is descreasing for $k>\lambda - 1$. The maximum value is at a $k$ such that $\lambda-1<k<\lambda$. As $k$ is an integer, it must be $\lfloor\lambda\rfloor$ (as we assumed that $\lambda$ is not an integer).

  If $\lambda$ is an integer, by a similar argument, we can show that $i\mapsto P\{X=i\}$ is non-decreasing for $k\leq\lambda$ and non-increasing for $k\geq\lambda-1$. Then the maximum value is at $k$ such that $\lambda-1\leq k\leq \lambda$. Since both $k$ and $\lambda$ are integers, the same maximum value is at both $k=\lambda$ and $k=\lambda -1$.
\end{solution}
