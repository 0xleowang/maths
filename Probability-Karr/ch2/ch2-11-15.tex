% Exercise 2.11
\begin{exercise}
  a) Let $X$ have distribution function $F$. For each $k$, calculate the distribution function of $|X|^k$. \\
  b) Let $X$ be absolutely continuous. Compute the density of $|X|$.
\end{exercise}
\begin{solution}
  a) If $k>0$,
  \begin{align*}
    P(|X|^k\leq t) &= P(|X|\leq t^{1/k}) \\
      &= P(-t^{1/k}\leq X\leq t^{1/k}) \\
      &= P(X\leq t^{1/k}) - P(X<-t^{1/k}) \\
      &= F(t^{1/k}) - F((-t^{1/k})-) .
  \end{align*}
  If $k<0$,
  \begin{align*}
    P(|X|^k\leq t) &= P(|X|\geq t^{1/k}) \\
      &= P(\{X\geq t^{1/k}\}\cup\{X\leq -t^{1/k}\}) \\
      &= P(X\geq t^{1/k}) + P(X\leq -t^{1/k}) \quad \mbox{as disjoint} \\
      &= 1 - P(X<t^{1/k}) + P(\leq -t^{1/k}) \\
      &= 1 - F(t^{1/k}-) + F(-t^{1/k}) .
  \end{align*}
  If $k=0$, $P(|X|^0\geq t) = P(1\leq t)$. Since $\{1\leq t\}=\emptyset, \forall t<1$ and $\{1\leq t\}=\Omega, \forall t\geq 1$,
  \[
    P(1\leq t) = \begin{cases}
      0 & t < 1 \\
      1 & t \geq 1
    \end{cases},
  \]
  which is a step function. \\

  b) \begin{align*}
    F_{|X|}(t) = P(|X|\leq t) &= P(-t\leq X\leq t) \\
      &= P(X\leq t) - P(X<-t) \\
      &= F_X(t) - F_X((-t)-) \\
      &= F_X(t) - F_X(-t) .
  \end{align*}
  The last equality is justified as $X$ is absolutely continuous. Differentiating the above equation with respect to $t$ on both sides gives the density
  \[ f_{|X|}(t) = f_X(t) + f_X(-t) .\]
\end{solution}


% Exercise 2.12
\begin{exercise}
  Let $X$ have distribution function $F$. Calculate the distribution functions of $X^+$ and $X^-$.
\end{exercise}
\begin{solution}
  For $X^+$, if $t<0$, then $\{X^+\leq t\}=\emptyset$ and $P(X^+\leq t)=0$. If $t\geq 0$, then $\{X^+\leq t\} = \{0<X\leq t\}+\{X\leq 0\}$. Hence
  \begin{align*}
    P(X^+\leq t) &= P(0<X\leq t) + P(X\leq 0) \\
      &= P(X\leq t) - P(X\leq 0) + P(X\leq 0) \\
      &= P(X\leq t) \\
      &= F(t) .
  \end{align*}
  Combining both cases, $F_{X^+}(t)=P(X^+\leq t) = F(t)\dsone\{t\geq 0\}$. \\

  For $X^-$, $P(X^-\leq t) = 1 - P(X^->t)$. If $t\geq 0$, then $\{X^->t\}=\emptyset$ $P(X^t>t) = 0$. If $t<0$, then $\{X^->t\} = \{t<X\leq 0\}+\{X>0\}$. Hence
  \begin{align*}
    P(X^->t) &= P(t<X\leq 0) + P(X> 0) \\
      &= P(X\leq 0) - P(X\leq t) + 1 - P(X\leq 0) \\
      &= 1 - P(X\leq t) \\
      &= 1 - F(t) .
  \end{align*}
  Combining both cases, $F_{X^-}(t)=P(X^-\leq t) = \dsone\{t\geq 0\} + (1-\dsone\{t\geq 0\})F(t)$.
\end{solution}


% Exercise 2.13
\begin{exercise}
  Assume that $X$ is positive and absolutely continuous with density $f$ and distribution $F$ satisfying $F(s)<1$ for all $s<\infty$, and let $H(s)=-\log(1-F(s))$. \\
  a) Prove that $H$ is differentiable with derivative $h=f/(1-F)$, which is termed the \textit{hazard function} of $T$ (or $F$). \\
  b) Prove that for each $t$,
  \[ h(t) = \lim_{h\downarrow 0}\frac{1}{h}P\{X\leq t+h|X>t\}. \]
  c) Prove that $\int_0^\infty h(t)dt=\infty$. \\
  d) Prove that
  \[ P\{X>t+s|X>t\} = \exp\bigg[-\int_t^{t+s}h(u)du \bigg] \]
  for each $t$ and $s$. \\
  e) Prove that $X\deq E(\lambda)$ if and only if $h\equiv \lambda$.
\end{exercise}
\begin{solution}
  a) Since $F(s)<1$ for all $s<\infty$, we have $1-F(s)>0$. As $\log(x)$ is continuous and differentiable for all $x>0$, it is clear that $H$ is differentiable. Applying chain rule, it can be easily showed that $h=f/(1-F)$. \\

  b) \begin{align*}
    \lim_{h\downarrow 0}\frac{1}{h}P(X\leq t+h|X>t) &= \lim_{h\downarrow 0}\frac{1}{h}\frac{P(t<x\leq t+h)}{P(x>t)} \\
      &= \lim_{h\downarrow 0}\frac{1}{h}\frac{P(x\leq t+h) - P(x\leq t)}{1-P(X\leq t)} \\
      &= \lim_{h\downarrow 0}\frac{1}{h}\frac{F(t+h)-F(t)}{1-F(t)} \\
      &= \frac{1}{1-F(t)}\lim_{h\downarrow 0} \frac{F(t+h)-f(t)}{h} \\
      &= \frac{f(t)}{1-F(t)} \\
      &= h(t).
  \end{align*}

  c) As $X$ is positive, $F(0)=0$. Then,
  \begin{align*}
    \int_0^\infty h(t)dt &= \lim_{n\to\infty} H(n) - H(0) \\
      &= \lim_{n\to\infty} -\log(1-F(n)) + \log(1-F(0)) \\
      &= \lim_{n\to\infty} -\log(1-F(n)) .
  \end{align*}
  Given $F(n)\to 1$ as $n\to\infty$, it is clear that $\int_0^\infty h(t)dt=\infty$.

  d) \begin{align*}
    P(X>t+s|X>t) &= \frac{P(X>t+s)}{P(X>t)} \\
      &= \frac{1 - P(X\leq t+s)}{1 - P(X\leq t)} \\
      &= \frac{1 - F(t+s)}{1 - F(t)} \\
      &= \frac{\exp\{-H(t+s)\}}{\exp\{-H(t)\}} \\
      &= \exp\{-(H(t+s)-H(t))\} \\
      &= \exp\bigg[ -\int_t^{t+s}h(u)du \bigg] .
  \end{align*}

  e) For sufficiency, if $X\deq E(\lambda)$, we have $f(x)=\lambda e^{=\lambda x}, x>0$ and $F(x)=1-e^{-\lambda x}, x>0$. Thus,
  \[h(x) = \frac{f(x)}{1-F(x)} = \frac{\lambda e^{-\lambda x}}{e^{-\lambda x}} = \lambda, \quad \forall x>0 .\]
  For necessity, if $h\equiv \lambda$, we have $\lambda = f/(1-F)$. This is equivalent to the following ODE (ordinary differential equation):
  \[ \pderiv{F}{x} + \lambda F = \lambda . \]
  Taking derivatives of $e^{\lambda x}F$ gives
  \[ \pderiv{(e^{\lambda x}F(x))}{x} = e^{\lambda x}\pderiv{F(x)}{x} + \lambda e^{\lambda x}F(x) = \lambda e^{\lambda x}.\]
  The last equality is obtained by using the above ODE. Therefore, we have
  \[ e^{\lambda x}F(x) = e^{\lambda x} + C \quad\Rightarrow\quad F(x) = 1 + Ce^{-\lambda x} ,\]
  where $C$ is a constant. Since $X$ is positive and $F(0)=0$, we have $C=-1$ and $F(x)=1-e^{-\lambda x}$ that is the distribution function of $E(\lambda)$.
\end{solution}
