% Exercise 2.11
\begin{exercise}
  a) Let $X$ have distribution function $F$. For each $k$, calculate the distribution function of $|X|^k$. \\
  b) Let $X$ be absolutely continuous. Compute the density of $|X|$.
\end{exercise}
\begin{solution}
  a) If $k>0$,
  \begin{align*}
    P\{|X|^k\leq t\} &= P\{|X|\leq t^{1/k}\} \\
      &= P\{-t^{1/k}\leq X\leq t^{1/k}\} \\
      &= P\{X\leq t^{1/k}\} - P\{X<-t^{1/k}\} \\
      &= F(t^{1/k}) - F((-t^{1/k})-) .
  \end{align*}
  If $k<0$,
  \begin{align*}
    P\{|X|^k\leq t\} &= P\{|X|\geq t^{1/k}\} \\
      &= P\{\{X\geq t^{1/k}\}\cup\{X\leq -t^{1/k}\}\} \\
      &= P\{X\geq t^{1/k}\} + P\{X\leq -t^{1/k}\} \quad \mbox{as disjoint} \\
      &= 1 - P\{X<t^{1/k}\} + P\{\leq -t^{1/k}\} \\
      &= 1 - F(t^{1/k}-) + F(-t^{1/k}) .
  \end{align*}
  If $k=0$, $P\{|X|^0\geq t\} = P\{1\leq t\}$. Since $\{1\leq t\}=\emptyset, \forall t<1$ and $\{1\leq t\}=\Omega, \forall t\geq 1$,
  \[
    P\{1\leq t\} = \begin{cases}
      0 & t < 1 \\
      1 & t \geq 1
    \end{cases},
  \]
  which is a step function. \\

  b) \begin{align*}
    F_{|X|}(t) = P\{|X|\leq t\} &= P\{-t\leq X\leq t\} \\
      &= P\{X\leq t\} - P\{X<-t\} \\
      &= F_X(t) - F_X((-t)-) \\
      &= F_X(t) - F_X(-t) .
  \end{align*}
  The last equality is justified as $X$ is absolutely continuous. Differentiating the above equation with respect to $t$ on both sides gives the density
  \[ f_{|X|}(t) = f_X(t) + f_X(-t) .\]
\end{solution}


% Exercise 2.12
\begin{exercise}
  Let $X$ have distribution function $F$. Calculate the distribution functions of $X^+$ and $X^-$.
\end{exercise}
\begin{solution}
  For $X^+$, if $t<0$, then $\{X^+\leq t\}=\emptyset$ and $P\{X^+\leq t\}=0$. If $t\geq 0$, then $\{X^+\leq t\} = \{0<X\leq t\}+\{X\leq 0\}$. Hence
  \begin{align*}
    P\{X^+\leq t\} &= P\{0<X\leq t\} + P\{X\leq 0\} \\
      &= P\{X\leq t\} - P\{X\leq 0\} + P\{X\leq 0\} \\
      &= P\{X\leq t\} \\
      &= F(t) .
  \end{align*}
  Combining both cases, $F_{X^+}(t)=P\{X^+\leq t\} = F(t)\dsone\{t\geq 0\}$. \\

  For $X^-$, $P\{X^-\leq t\} = 1 - P\{X^->t\}$. If $t\geq 0$, then $\{X^->t\}=\emptyset$ $P\{X^t>t\} = 0$. If $t<0$, then $\{X^->t\} = \{t<X\leq 0\}+\{X>0\}$. Hence
  \begin{align*}
    P\{X^->t\} &= P\{t<X\leq 0\} + P\{X> 0\} \\
      &= P\{X\leq 0\} - P\{X\leq t\} + 1 - P\{X\leq 0\} \\
      &= 1 - P\{X\leq t\} \\
      &= 1 - F(t) .
  \end{align*}
  Combining both cases, $F_{X^-}(t)=P\{X^-\leq t\} = \dsone\{t\geq 0\} + (1-\dsone\{t\geq 0\})F(t)$.
\end{solution}


% Exercise 2.13
\begin{exercise}
  Assume that $X$ is positive and absolutely continuous with density $f$ and distribution $F$ satisfying $F(s)<1$ for all $s<\infty$, and let $H(s)=-\log(1-F(s))$. \\
  a) Prove that $H$ is differentiable with derivative $h=f/(1-F)$, which is termed the \textit{hazard function} of $T$ (or $F$). \\
  b) Prove that for each $t$,
  \[ h(t) = \lim_{h\downarrow 0}\frac{1}{h}P\{X\leq t+h|X>t\}. \]
  c) Prove that $\int_0^\infty h(t)dt=\infty$. \\
  d) Prove that
  \[ P\{X>t+s|X>t\} = \exp\bigg[-\int_t^{t+s}h(u)du \bigg] \]
  for each $t$ and $s$. \\
  e) Prove that $X\deq E(\lambda)$ if and only if $h\equiv \lambda$.
\end{exercise}
\begin{solution}
  a) Since $F(s)<1$ for all $s<\infty$, we have $1-F(s)>0$. As $\log(x)$ is continuous and differentiable for all $x>0$, it is clear that $H$ is differentiable. Applying chain rule, it can be easily showed that $h=f/(1-F)$. \\

  b) \begin{align*}
    \lim_{h\downarrow 0}\frac{1}{h}P\{X\leq t+h|X>t\} &= \lim_{h\downarrow 0}\frac{1}{h}\frac{P\{t<X\leq t+h\}}{P\{X>t\}} \\
      &= \lim_{h\downarrow 0}\frac{1}{h}\frac{P\{X\leq t+h\} - P\{X\leq t\}}{1-P\{X\leq t\}} \\
      &= \lim_{h\downarrow 0}\frac{1}{h}\frac{F(t+h)-F(t)}{1-F(t)} \\
      &= \frac{1}{1-F(t)}\lim_{h\downarrow 0} \frac{F(t+h)-f(t)}{h} \\
      &= \frac{f(t)}{1-F(t)} \\
      &= h(t).
  \end{align*}

  c) As $X$ is positive, $F(0)=0$. Then,
  \begin{align*}
    \int_0^\infty h(t)dt &= \lim_{n\to\infty} H(n) - H(0) \\
      &= \lim_{n\to\infty} -\log(1-F(n)) + \log(1-F(0)) \\
      &= \lim_{n\to\infty} -\log(1-F(n)) .
  \end{align*}
  Given $F(n)\to 1$ as $n\to\infty$, it is clear that $\int_0^\infty h(t)dt=\infty$.

  d) \begin{align*}
    P\{X>t+s|X>t\} &= \frac{P\{X>t+s\}}{P\{X>t\}} \\
      &= \frac{1 - P\{X\leq t+\})}{1 - P\{X\leq t\}} \\
      &= \frac{1 - F(t+s)}{1 - F(t)} \\
      &= \frac{\exp\{-H(t+s)\}}{\exp\{-H(t)\}} \\
      &= \exp\{-(H(t+s)-H(t))\} \\
      &= \exp\bigg[ -\int_t^{t+s}h(u)du \bigg] .
  \end{align*}

  e) For sufficiency, if $X\deq E(\lambda)$, we have $f(x)=\lambda e^{=\lambda x}, x>0$ and $F(x)=1-e^{-\lambda x}, x>0$. Thus,
  \[h(x) = \frac{f(x)}{1-F(x)} = \frac{\lambda e^{-\lambda x}}{e^{-\lambda x}} = \lambda, \quad \forall x>0 .\]
  For necessity, if $h\equiv \lambda$, we have $\lambda = f/(1-F)$. This is equivalent to the following ODE (ordinary differential equation):
  \[ \pderiv{F}{x} + \lambda F = \lambda . \]
  Taking derivatives of $e^{\lambda x}F$ gives
  \[ \pderiv{(e^{\lambda x}F(x))}{x} = e^{\lambda x}\pderiv{F(x)}{x} + \lambda e^{\lambda x}F(x) = \lambda e^{\lambda x}.\]
  The last equality is obtained by using the above ODE. Integrating on both sides of the above equation, we have
  \[ e^{\lambda x}F(x) = e^{\lambda x} + C \quad\Rightarrow\quad F(x) = 1 + Ce^{-\lambda x} ,\]
  where $C$ is a constant. Since $X$ is positive and $F(0)=0$, we have $C=-1$ and $F(x)=1-e^{-\lambda x}$ that is the distribution function of $E(\lambda)$.
\end{solution}


% Exercise 2.14
\begin{exercise}
  a) Prove that if $X$ has a geometric distribution then
  \begin{equation}\label{eq:2.15}
    P\{X>n+k|X>n\} = P\{X>k\}
  \end{equation}
  for each $n$ and $k$. \\
  b) Prove that if $X$ is a positive and integer-valued and satisfies \eqref{eq:2.15}, then $X$ has a geometric distribution.
\end{exercise}
\begin{solution}
  a) If $X$ has a geometric distribution, then $P\{X=k\}=p(1-p)^{k-1}, k\geq 1$ for some $p\in(0, 1)$. For an arbitrary $n$ and $k$, the LHS of \eqref{eq:2.15} has
  \begin{align*}
    P\{X>n+k|X>n\} &= \frac{P\{X>n+k\}}{P\{X>n\}} \\
      &= \frac{\Sigma_{i=n+k+1}^\infty p(1-p)^{i-1}}{\Sigma_{i=n+1}^\infty p(1-p)^{i-1}} \\
      &= \frac{p(1-p)^{n+k}\Sigma_{j=0}^\infty(1-p)^j}{p(1-p)^n\Sigma_{j=0}^\infty(1-p)^j} \\
      &= (1-p)^k.
  \end{align*}
  Then, the RHS of \eqref{eq:2.15} has
  \[
    P\{X>k\} = \Sigma_{i=k+1}^\infty p(1-p)^{i-1} = p(1-p)^k\Sigma_{j=0}^\infty(1-p)^j = \frac{p(1-p)^k}{1-(1-p)} = (1-p)^k.
  \]
  As the LHS equals to the RHS, the proof is compelte. \\

  b) Applying the \textit{memoryless} property for $k=1$, we can get
  \begin{align*}
    1 - P\{X=1\} = P\{X>1\} &= P\{X>n+1|X>n\} \\
      &= \frac{P\{X>n+1\}}{P\{X>n\}} \\
      &= \frac{1 - P\{X\leq n+1\}}{1 - P\{X\leq n\}} \\
      &= \frac{1 - P\{X\leq n\} - P\{X= n+1\}}{1 - P\{X\leq n\}} \\
      &= 1 - \frac{P\{X= n+1\}}{1 - P\{X\leq n\}} .
  \end{align*}
  Rearranging gives
  \[ P\{X=n+1\} = P\{X=1\}(1-P\{X\leq n\}) .\]
  Denote $P\{X=1\}$ by $p$. For $n=0$,
  \[ P\{X=1\} = P\{X=1\}(1-P\{X\leq 0\}) = p .\]
  For $n=1$,
  \[ P\{X=2\} = P\{X=1\}(1-P\{X\leq 1\}) = p(1-p) .\]
  For $n=2$,
  \begin{align*}
    P\{X=3\} &= P\{X=1\}(1-P\{X\leq 2\}) \\
      &= P\{X=1\}(1-P\{X=1\}-P\{X=2\}) \\
      &= p(1-p-p(1-p)) \\
      &= p(1-p)^2.
  \end{align*}
  Assume $P\{X=k\}=p(1-p)^{k-1}$ for all $k\leq n$, then for the $n+1$ case, we have
  \begin{align*}
    P\{X=n+1\} &= p(1-P\{X\leq n\}) \\
      &= p(1 - \Sigma_{k=1}^n p(1-p)^k) \\
      &= p(1 - p\Sigma_{k=1}^n (1-p)^k) \\
      &= p(1 - p \frac{1-(1-p)^n}{1-(1-p)}) \\
      &= p(1-p)^n .
  \end{align*}
  Therefore, we have shown that $P\{X=n\} = p(1-p)^{n-1}$ for all $n\geq 1$ by mathematical induction, and $X$ has a geometric distribution.
\end{solution}


% Exercise 2.15
\begin{exercise}
  a) Prove that if $Y\deq E(\lambda)$, then for all $t$ and $s$,
  \begin{equation}\label{eq:2.16}
    P\{Y>t+s|Y>t\} = P\{Y>s\}.
  \end{equation}
  b) Prove that if $Y$ is absolutely continuous, positive and satisfies \eqref{eq:2.16}, then $Y$ has an exponential distribution.
\end{exercise}
\begin{solution}
  a) If $Y\deq E(\lambda)$, then it has the distribution function $F(x) = 1-e^{-\lambda x}, x>0$.
  The LHS of \eqref{eq:2.16} has
  \begin{align*}
    P\{Y>t+s|Y>t\} &= \frac{P\{Y>t+s\}}{P\{Y>t\}} \\
      &= \frac{1-P\{Y\leq t+s\}}{1-P\{Y\leq t\}} \\
      &= \frac{1-F(t+s)}{1-F(t)} \\
      &= \frac{e^{-\lambda (t+s)}}{e^{-\lambda t}} \\
      &= e^{-\lambda s} .
  \end{align*}
  And the RHS of \eqref{eq:2.16} has
  \[ P\{X>s\} = 1-P\{X\leq s\} = 1-F(s) = e^{-\lambda s} ,\]
  which is equal to the LHS. \\

  b) Let $S(t)=P\{Y>t\}$ be the survivor function of $Y$. Using \eqref{eq:2.16}, we can get
  \[ S(t+s) = S(t)S(s), \quad \forall s,t>0 .\]
  This means that $S(t)$ must be in the exponential form, which has the general form $S(t)=ae^{bt}, a,b\in\bR$. Given $X$ is positive and hence $S(0)=1$, we can get $a=1$. Since $S(t)\to 0$ as $t\to\infty$, $b$ must be negative. Let $\lambda=-b$, thus $\lambda>0$ and $Y$ has the distribution function $F(x) = 1 - S(x) = 1 - e^{-\lambda x}$ (the same distribution function of $E(\lambda)$).
\end{solution}
