% Exercise 2.16
\begin{exercise}
  Let $X$ have distribution $N(0,1)$. Calculate the density of $Y=e^X$, which is said to have a \textit{log normal distribution}.
\end{exercise}
\begin{solution}
  We can write $Y=e^X$ as $Y=g(X)$ where $g(x):=e^x$. Then $g$ has an inverse function $h(y)=\log y$ that has derivative $h'(y)=1/y$. By Theorem 2.41, we have
  \[
    f_Y(y)=f_X(h(y))|h'(t)| = f_X(\log y)/y = \frac{1}{\sqrt{2\pi}y}e^{-(\log y)^2/2}, \quad y>0.
  \]
\end{solution}


% Exercise 2.17
\begin{exercise}
  Let $Y=g(X)$, where $X$ is a random variable and $g:\bR\to\bR$ is Borel measurable. Prove tha t$\sigma(Y)\subseteq\sigma(X)$. Conclude that if also $X=h(Y)$ for some $h$, then $\sigma(X) =\sigma(Y)$.
\end{exercise}
\begin{solution}
  As $g$ is Borel measurable, $g^{-1}(B)\in\cB(\bR)$ for every $B\in\cB(\bR)$. Thus,
  \begin{align*}
    \sigma(Y) &= \{Y^{-1}(B):B\in\cB(\bR)\} \\
      &= \{X^{-1}(g^{-1}(B)):B\in\cB(\bR)\} \\
      &\subseteq \{X^{-1}(B):B\in\cB(\bR)\} = \sigma(X).
  \end{align*}
  If $X=h(Y)$ for some Borel measurable function $h$, then $\sigma(X)\subseteq \sigma(Y)$, and hence $\sigma(X)=\sigma(Y)$.
\end{solution}


% Exercise 2.18
\begin{exercise}
  Prove that if $U\deq U[0, 1]$, and $\lambda>0$, then
  \[ X = -(1/\lambda)\log U \]
  satisfies $X\deq E(\lambda)$.
\end{exercise}
\begin{solution}
  Let $g(u)=-(1/\lambda)\log u$, then $X=g(U)$. As $g(u)$ is monotonically descreasing in $u>0$, it has an inverse function $h(x)=e^{-\lambda x}$ with the derivative $h'(x)=-\lambda E^{-\lambda x}$. Since $U\in(0, 1]$, $X$ has values in $[0, \infty)$. Applying the Theorem 2.41,
  \begin{align*}
    f_X(x) = f_U(h(x))|h'(x)| = \lambda e^{-\lambda x}, \quad x\geq 0,
  \end{align*}
  which has the same density function as $E(\lambda)$. Hence their distribution functions must be the same and $X\deq E(\lambda)$.
\end{solution}


% Exercise 2.19
\begin{exercise}
  Calculate the density function of $Y=1/X - X$, where $X\deq U[0, 1]$.
\end{exercise}
\begin{solution}
  Solve $y=1/x - x$ and write $x$ in terms of $y$, we can get $X = -\frac{Y}{2} + \demi\sqrt{4+Y^2}$ or $X = -\frac{Y}{2} - \demi\sqrt{4+Y^2}$. As $X\in[0, 1]$, only $X = -\frac{Y}{2} + \demi\sqrt{4+Y^2}$ is valid. Moreover, $Y$ has values in $[0, \infty)$. Let $h(y)=-\frac{y}{2} + \demi\sqrt{4+y^2}$, then $X=h(Y)$. Given $h'(y)=-\demi + \frac{y}{2\sqrt{4+y^2}}$, applying Theorem 2.41, we can get the density of $Y$:
  \[
    f_Y(y) = f_U(h(y))|h'(y)| = 1\cdot |-\demi+\frac{y}{2\sqrt{4+y^2}}| = \demi - \frac{y}{2\sqrt{4+y^2}}, \quad y\geq 0.
  \]
  (The answer can be verified by checking $\int_0^\infty f_Y(y)dy = 1$.)
\end{solution}
