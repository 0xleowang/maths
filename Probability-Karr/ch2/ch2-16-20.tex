% Exercise 2.16
\begin{exercise}
  Let $X$ have distribution $N(0,1)$. Calculate the density of $Y=e^X$, which is said to have a \textit{log normal distribution}.
\end{exercise}
\begin{solution}
  We can write $Y=e^X$ as $Y=g(X)$ where $g(x):=e^x$. Then $g$ has an inverse function $h(y)=\log y$ that has derivative $h'(y)=1/y$. By Theorem 2.41, we have
  \[
    f_Y(y)=f_X(h(y))|h'(t)| = f_X(\log y)/y = \frac{1}{\sqrt{2\pi}y}e^{-(\log y)^2/2}, \quad y>0.
  \]
\end{solution}


% Exercise 2.17
\begin{exercise}
  Let $Y=g(X)$, where $X$ is a random variable and $g:\bR\to\bR$ is Borel measurable. Prove tha t$\sigma(Y)\subseteq\sigma(X)$. Conclude that if also $X=h(Y)$ for some $h$, then $\sigma(X) =\sigma(Y)$.
\end{exercise}
\begin{solution}
  As $g$ is Borel measurable, $g^{-1}(B)\in\cB(\bR)$ for every $B\in\cB(\bR)$. Thus,
  \begin{align*}
    \sigma(Y) &= \{Y^{-1}(B):B\in\cB(\bR)\} \\
      &= \{X^{-1}(g^{-1}(B)):B\in\cB(\bR)\} \\
      &\subseteq \{X^{-1}(B):B\in\cB(\bR)\} = \sigma(X).
  \end{align*}
  If $X=h(Y)$ for some Borel measurable function $h$, then $\sigma(X)\subseteq \sigma(Y)$, and hence $\sigma(X)=\sigma(Y)$.
\end{solution}

