% Exercise 2.1
\begin{exercise}
  Let $\Omega=\{1,2,3,4,5,6\}$, and let$\cF=\sigma(\{1,2,3,4\},\{3, 4, 5, 6\})$. \\
  a) List all sets in $\cF$. \\
  b) Is the function
    \[
      X(\omega) = \begin{cases}
        2 & \omega = 1, 2, 3, 4 \\
        7 & \omega = 5, 6
      \end{cases}
    \]
    a random variable over $(\Omega, \cF)$? \\
  c) Given an example of a function on $\Omega$ that is \textit{not} a random variable over $(\Omega, \cF)$. \\
  d) Show that there exists a probability $P$ on $(\Omega, \cF)$ such that $P(A)$ is zero or one for all $A\in\cF$, yet $P$ is not a point mass.
\end{exercise}
\begin{solution}
  a) \[ \cF = \{ \emptyset, \{1, 2\}, \{3, 4\}, \{5, 6\}, \{1, 2, 3, 4\}, \{1, 2, 5, 6\}, \{3, 4, 5, 6\}, \Omega \} .\]

  b) \[
       \{X\leq t\} = \begin{cases}
         \emptyset & t < 2 \\
         \{1, 2, 3, 4\} & 2 \leq t < 7 \\
         \Omega & t \geq 7
       \end{cases}
     \]
    Given $\{X\leq t\}$ for every $t\in\bR$, by Proposition 2.12, $X$ is a random variable over $(\Omega, \cF)$. \\

  c) Consider a function $Y$ such that
    \[
      Y(\omega) = \begin{cases}
        2 & \omega = 1 \\
        7 & \omega = 2, 3, 4, 5, 6
      \end{cases}
    \]
    Since $\{X\leq 2\} = \{1\}\notin\cF$, $Y$ is not a random variable over $(\Omega, \cF)$. \\

  d) Consider a probability meausre $P$ such that $P(A) = \dsone(A\supseteq\{1, 2\})$. It can be easily verified that $P$ is a probability measure, yet $P$ is not a point mass.

\end{solution}


% Exercise 2.2
\begin{exercise}
  Prove that if $X$ and $Y$ are random variables, then $\{X\leq Y\}$, $\{X<Y\}$ and $\{X=Y\}$ are events.
\end{exercise}
\begin{solution}
  To prove $\{X\leq Y\}$ is an event, we can show that $\{X\leq Y\}\in\cF$. By Proposition 2.13, $X-Y$ is a random variable. By Proposition 2.12, $\{X\leq Y\}=\{X-Y\leq 0\}\in\cF$. Similarly, $\{X<Y\}\in\cF$. Hence $\{X=Y\}=\{X\leq Y\}\backslash\{X<Y\}\in\cF$.
\end{solution}


% Exercise 2.3
\begin{exercise}
  Let $X$ and $Y$ be random variables and let $A$ be an event. Prove that the function
  \[
    Z(\omega) = \begin{cases}
      X(\omega) & \mbox{if } \omega\in A \\
      Y(\omega) & \mbox{if } \omega\in A^c
    \end{cases}
  \]
  is a random variable.
\end{exercise}
\begin{solution}
  Recall that indicator functions of an event are random variables (Example 2.4). As $Z = \dsone_A X + (1-\dsone_A) Y$, by Proposition 2.13, $Z$ is a random variable.
\end{solution}


% Exercise 2.4
\begin{exercise}
  Let $\cG=\{A_1,\ldots,A_n\}$ be a finite partition of $\Omega$, and let $\cF=\sigma(\cG)$. \\
  a) Prove that a function $X:\Omega\to\bR$ is a random variable if and only if $X$ is constant over each partition set $A_i$. \\
  b) Use part a) to show that provided $\cF\neq \cP(\Omega)$, there exist functions $Y$ on $\Omega$ such that $|Y|$ is a random variable but $Y$ is not.
\end{exercise}
\begin{solution}
  [TODO]
\end{solution}


% Exercise 2.5
\begin{exercise}
  Let $X^+$ and $X^-$ be the positive and negative parts of the function $X:\Omega\to\bR$. Prove that $X=X^+-X^-$ and $|X|=X^+ + X^-$.
\end{exercise}
\begin{solution}
  For $\omega\in\Omega$ such that $X(\omega)\geq 0$, $X^+(\omega) = X$ and $X^-(\omega)=0$. Then, $X = X - 0 = X^+ - X^-$ and $|X| = X + 0 = X^+ + X^-$. For $\omega\in\Omega$ such that $X(\omega)< 0$, $X^+(\omega)=0$ and $X^-(\omega) = -X$. Then, $X = 0 - (-X) = X^+ - X^-$ and $|X| = 0 + (-X) = X^+ + X^-$. As the equations hold for all $\omega\in\Omega$, the proof is complete.
\end{solution}
